\documentclass[a4paper,10pt]{article}
\usepackage[left=2.5cm,right=2.5cm,top=3cm,bottom=3cm]{geometry}
\usepackage[utf8x]{inputenc}
\usepackage[ngerman]{babel}
\usepackage[T1]{fontenc} % T1-encoded fonts: Auch Wörter mit Umlauten trennen
\usepackage{colortbl}    % Für farbige Zeilen/Reihen in Tabellen
\usepackage{xcolor}      % Definitionen von Farben
\thispagestyle{empty}    % Keine Kopf- und Fusszeile

% Beginn des Dokuments
\begin{document}
  \section*{\small{Software Engineering II \\ Wintersemester 2014/2015: Gruppe 2.3 \\ Katja Kirstein}}
  \section*{Testprotokoll 2.12.14}
  Verwendete Hardware Geme 8 mit Festo 5 als Band 1 und Geme 7 mit Festo 6 als Band 2.\\
  Alle Tests wurden bestanden, ohne Blinken der Ampel und Button LEDs, da diese zu diesem Stand noch nicht implementiert wurden.
  \newline
\begin{enumerate}
 \item  Puks fahren in abwechselnder Reihenfolge bis ans Ende von Band 2
    \begin{itemize}
    \item Metall, Nicht Metall
    \item Nicht Metall, Metall
   \end{itemize}
 \item Puks die außerhalb des Toleranzbereichs werden auf Band 1 mit Hilfe der Weiche aussortiert
 \item Ein Puk der falschherum auf Band 1 gelegt wird, fährt bis ans Ende von Band 1, wartet, wird umgedreht,wieder gestartet 
    und durchläuft Band 2 bis zum Ende
 \item Ein Puk der falschherum am Ende von Band 1 ankommt, wird nicht umgedreht und auf Band 2 durch die Weiche aussortiert
 \item Wenn auf Band 2 falsche Reihenfolge der Puks ankommt fährt der 2 Puk nach links bis zum Anfang von Band 2
   \begin{itemize}
    \item Metall, Metall
    \item Nicht Metall, Nicht Metall
   \end{itemize}
 \item Mehrere Puks auf Band 1, sowie nur ein Puk auf Band 2. Band 1 wartet sobald der nächste Puk am Ende des Bandes ist, bis Band 2 wieder frei ist 
\end{enumerate}


\end{document}