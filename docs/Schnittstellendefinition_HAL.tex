\documentclass[a4paper,10pt]{article}\\
\usepackage[left=2.5cm,right=2.5cm,top=3cm,bottom=3cm]{geometry}
\usepackage[utf8x]{inputenc}
\usepackage[ngerman]{babel}
\usepackage[T1]{fontenc} % T1-encoded fonts: auch Wörter mit Umlauten trennen
\thispagestyle{empty}    % Keine Kopf- und Fusszeile

% Beginn des Dokuments
\begin{document}
  \section*{Schnittstellendefinition der HAL}
  Die Aufgabe der HAL ist es hardwarezugriffe zu abstrahieren. Zugriffe auf Hardware nur über HAL möglich.
  Der zur HAL gehörige Header ist HWaccess.hpp. Alle Module die hardwarezugriff benötigen müssen den HAL-header inkludieren. \newline
   
  \textbf{\newline Port A}
  \begin{itemize}
    \item enum motor\_mode(stop, right, left)
    \item void set\_motor(enum motor\_mode, bool slow) 
    \item void open\_switch()
    \item void close\_switch()
    \item enum light\_color(red, yellow, grenn)
    \item void set\_light(enum light\_color, bool on) 
  \end{itemize}
  
  \textbf{\newline Port B}
  \begin{itemize}
	\item enum light\_barriers(entrance\_sensor, height\_sensor, switch\_sensor, light\_sensor, exit\_sensor)
    \item bool obj\_in\_light\_barrier(enum light\_barriers)
    \item bool obj\_has\_valid\_height()
    \item bool obj\_has\_metall()
    \item bool is\_switch\_open()
  \end{itemize}
  
  \textbf{\newline Port C}
  \begin{itemize}
    \item enum button\_light(start\_button, reset\_button, q1\_button, q2\_button)
    \item void set\_led\_state(enum button\_light)
    \item enum button( start, stop, reset, e\_stop)
     \item bool is\_button\_pressed(enum button)
  \end{itemize}

\end{document}