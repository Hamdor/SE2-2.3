\documentclass[a4paper,10pt]{article}
\usepackage[left=2.5cm,right=2.5cm,top=3cm,bottom=3cm]{geometry}
\usepackage[utf8x]{inputenc}
\usepackage[ngerman]{babel}
\usepackage[T1]{fontenc} % T1-encoded fonts: Auch Wörter mit Umlauten trennen
\usepackage{colortbl}    % Für farbige Zeilen/Reihen in Tabellen
\usepackage{xcolor}      % Definitionen von Farben
\thispagestyle{empty}    % Keine Kopf- und Fusszeile

% Beginn des Dokuments
\begin{document}
  \section*{\small{Software Engineering II \\ Wintersemester 2014/2015: Gruppe 2.3 \\ Marian Triebe}}
  \section*{Protokoll der seriellen Schnittstelle (Version 2)}
  \textbf{\\ {\large Protokoll}}\\
  \newline
  \begin{tabular}{|l|l|}
    \hline
    \rowcolor{lightgray}\textbf{Wert} & \textbf{Beschreibung}\\
    \hline
    0x00 & ERR\_STOP - Band signalisiert Fehler\\
    \hline
    0x01 & ERR\_QUIT - Band signalisiert Quettierung des Fehlers\\
    \hline
    0x02 & RESUME - Band kann weiterlaufen (Start-Taste)\\
    \hline
    0x03 & B2\_FREE - Band 2 kann einen neuen Puck verarbeiten\\
    \hline
    0x04 & E\_STOP - E-Stopp wurde auf einem Band gedrückt\\
    \hline
    0x05 & STOP - Stopp Taster wurde auf einem Band gedrückt\\
    \hline
    0x06 & NOTHING - Keine Nachricht\\
    \hline
  \end{tabular}
  \newline \newline
  
  \textbf{\\ {\large Nachrichten-Typen}}\\
  \newline
  \begin{tabular}{|l|l|}
    \hline
    \rowcolor{lightgray}\textbf{Wert} & \textbf{Beschreibung}\\
    \hline
    0x00 & MSG - Nachricht ohne Daten-Block\\
    \hline
    0x01 & DATA - Nachricht mit Daten-Block\\
    \hline
    0x02 & ERR - Fehlerhafte Übertragung\\
    \hline
  \end{tabular}
  \newline \newline
 
  \textbf{\\ {\large Datenblock}}\\
  \newline
  \begin{tabular}{|l|p{1.4cm}|l|}
    \hline
    \rowcolor{lightgray}\textbf{Name} & \textbf\centering{Datentyp} & \textbf{Beschreibung}\\
    \hline
    m\_type & \centering int & Typ des Telegrams\\
    \hline
    m\_msg & \centering int & Nachricht\\
    \hline
    m\_id & \centering int & ID des Tokens dessen Daten übertragen werden\\
    \hline
    m\_height1 & \centering int & Höhenwert 1 des Tokens\\
    \hline
    m\_height2 & \centering int & Höhenwert 2 des Tokens (immer 0, da noch nicht bekannt)\\
    \hline
  \end{tabular}
\end{document}