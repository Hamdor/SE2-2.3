\documentclass[a4paper,10pt]{article}
\usepackage[left=2.5cm,right=2.5cm,top=3cm,bottom=3cm]{geometry}
\usepackage[utf8x]{inputenc}
\usepackage[ngerman]{babel}
\usepackage[T1]{fontenc} % T1-encoded fonts: Auch Wörter mit Umlauten trennen
\usepackage{colortbl}    % Für farbige Zeilen/Reihen in Tabellen
\usepackage{xcolor}      % Definitionen von Farben
\thispagestyle{empty}    % Keine Kopf- und Fusszeile

% Beginn des Dokuments
\begin{document}
  \section*{\small{Software Engineering II \\ Wintersemester 2014/2015: Gruppe 2.3 \\ Marian Triebe}}
  \section*{Protokoll der seriellen Schnittstelle}
  \textbf{\\ {\large Protokoll}}\\
  \newline
  \begin{tabular}{|l|l|}
    \hline
    \rowcolor{lightgray}\textbf{Wert} & \textbf{Beschreibung}\\
    \hline
    0xFF & Synchronisation\\
    \hline
    0x01 & Start der Daten\\
    \hline
    0x02 & Erfolgreiche Datenübertragung\\
    \hline
    0x03 & Fehlerhafte Datenübertragung\\
    \hline
    0x04 & Fehler einer Maschine (stoppt die andere Maschine)\\
    \hline
    0x05 & Fehler einer Maschine behoben (setzt Betrieb fort)\\
    \hline
  \end{tabular}
  \newline \newline
  \textbf{Anmerkungen:}
  \newline
  Während einer Datenübertragung kann keine Synchronisation erfolgen. Sollte der Empfänger eines Datenblocks mit "0x03" \space antworten,
  dann sendet der Sender wieder "0x01" \space und die Daten.\\
  
  \textbf{\\ {\large Header}}\\
  \newline
  \begin{tabular}{|l|p{1cm}|l|}
    \hline
    \rowcolor{lightgray}\textbf{Name} & \textbf\centering{Größe (Byte)} & \textbf{Beschreibung}\\
    \hline
    Version & \centering 1 & Protokoll Version / Gilt auch für Header\\
    \hline
    Größe & \centering 1 & Anzahl der Bytes, die im Header folgen\\
    \hline
  \end{tabular}
  \newline \newline
 
  \textbf{\\ {\large Datenblock}}\\
  \newline
  \begin{tabular}{|l|p{1cm}|l|}
    \hline
    \rowcolor{lightgray}\textbf{Name} & \textbf\centering{Größe (Byte)} & \textbf{Beschreibung}\\
    \hline
    ID & \centering 1 & ID des Tokens (sollte 0xFF erreicht werden, ost der Nachfolgende 0x00)\\
    \hline
    Typ & \centering 1 & Typ des Tokens (Kunststoff, Metall, etc.)\\
    \hline
    Höhenwert & \centering 2 & Wert der ersten Höhenmessung\\
    \hline
  \end{tabular}
\end{document}