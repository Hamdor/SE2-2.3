\documentclass[a4paper,10pt]{article}
\usepackage[left=2.5cm,right=2.5cm,top=3cm,bottom=3cm]{geometry}
\usepackage[utf8x]{inputenc}
\usepackage[ngerman]{babel}
\usepackage[T1]{fontenc} % T1-encoded fonts: Auch Wörter mit Umlauten trennen
\usepackage{colortbl}    % Für farbige Zeilen/Reihen in Tabellen
\usepackage{xcolor}      % Definitionen von Farben
\thispagestyle{empty}    % Keine Kopf- und Fusszeile

% Beginn des Dokuments
\begin{document}
  \section*{\small{\flushright Stand: \today\\ \flushleft Software Engineering II \\ Wintersemester 2014/2015: Gruppe 2.3}}
  \section*{Termine der Milestones und Liste der Arbeitspakete}
  1. Milestone (1. Praktikumstermin): \textbf{16.10.2014 \textbar \space (OK am 21.10.2014)}\\
  2. Milestone (2. Praktikumstermin): \textbf{23.10.2014 \textbar \space (OK am 24.10.2014)}\\
  3. Milestone (4. Praktikumstermin): \textbf{13.11.2014 \textbar \space (OK am 15.11.2014)}\\
  4. Milestone (5. Praktikumstermin): \textbf{27.11.2014 \textbar \space (OK am 27.11.2014)}\\
  5. Milestone (6. Praktikumstermin): \textbf{04.12.2014 \textbar \space (verschoben auf den 18.12.2014)}\\
  6. Milestone (7. Praktikumstermin): \textbf{18.12.2014}\\
  \textit{6. Milestone (8. Praktikumstermin): \textbf{15.01.2015 (Reserve)}}\\
  \begin{small}
    \begin{center}
      \begin{tabular}{|l|c|}
        \hline
        \rowcolor{lightgray}\textbf{Arbeitspaket} & \textbf{Dauer}\\
        \hline
        Meeting abhalten & 1,5 Stunden / Woche\\
        \hline
        Moderation \& Agenda planen & 0,5 Stunden / Woche\\
        \hline
        Protokollführung & 1,0 Stunden / Woche\\
        \hline
        RDD bearbeiten & 2,0 Stunden / Woche\\
        \hline
        Git-Repository Verwaltung & 1,0 Stunden / Woche\\
        \hline
        Code-Qualität sicherstellen & 1,0 Stunden / Woche\\
        \hline
        Debugging und Fehlerbehandlung & 30,0 Stunden\\
        \hline
        Testing & 16,0 Stunden\\
        \hline
        \rowcolor{lightgray}\multicolumn{2}{|l|}{\textbf{0. Milestone}}\\
        \hline
        Kick-Off Meeting abhalten & 1,0 Stunden\\
        \hline
        \rowcolor{lightgray}\multicolumn{2}{|l|}{\textbf{1. Milestone}}\\
        \hline
        Interface für HAL erstellen & 3,0 Stunden\\
        \hline
        Use Cases feststellen & 5,0 Stunden\\
        \hline
        Requirements feststellen & 5,0 Stunden\\
        \hline
        UML-Diagramme erstellen & 8,0 Stunden\\
        \hline
        Regressionstests planen & 0,5 Stunden\\
        \hline
        \rowcolor{lightgray}\multicolumn{2}{|l|}{\textbf{2. Milestone}}\\
        \hline
        Projektstrukturplan erstellen & 3,0 Stunden\\
        \hline
        HAL der Aktorik implementieren & 8,0 Stunden\\
        \hline
        Serielle Schnittstelle implementieren & 4,0 Stunden\\
        \hline
        Testprogramm für Aktorik und serielle Schnittstelle erstellen & 1,0 Stunden\\
        \hline
        \rowcolor{lightgray}\multicolumn{2}{|l|}{\textbf{3. Milestone}}\\
        \hline
        Projektstrukturplan fertigstellen & 3,0 Stunden\\
        \hline
        HAL der Sensorik implementieren (via ISRs und Pulse-Messages) & 8,0 Stunden\\
        \hline
        Anlagensteuerung mit Zustandsautomaten modellieren & 2,0 Stunden\\
        \hline
        Regressionstests implementieren & 7,5 Stunden\\
        \hline
        \rowcolor{lightgray}\multicolumn{2}{|l|}{\textbf{4. Milestone}}\\
        \hline
        Callback-Mechanismus für Sensorik implementieren (Reactor Pattern) & 6,0 Stunden\\
        \hline
        Testprogramm für Implementierung des Callback-Mechanismus erstellen & 1,0 Stunden\\
        \hline
        Zustandsautomaten der Anlagensteuerung implementieren & 8,0 Stunden\\
        \hline
        Testprogramm für Implementierung der Zustandsautomaten erstellen & 1,0 Stunden\\
        \hline
        \rowcolor{lightgray}\multicolumn{2}{|l|}{\textbf{5. Milestone}}\\
        \hline
        Ablauf über beide Förderbänder implementieren (ohne Ausnahmebehandlung) & 5,0 Stunden\\
        \hline
        Dokumentation des fehlerfreien Testablaufs mit allen Bauteilen erstellen & 2,0 Stunden\\
        \hline
        Timer für Ausnahmebehandlung implementieren & 5,0 Stunden\\
        \hline
        Timingverhalten zwischen HW- und BS-Timer im RDD diskutiert & 1,0 Stunden\\
        \hline
        \rowcolor{lightgray}\multicolumn{2}{|l|}{\textbf{6. Milestone}}\\
        \hline
        Ablauf über beide Förderbänder implementieren (inkl. Ausnahmebehandlung) & 40,0 Stunden\\
        \hline
        Bedienhandbuch für die Werkstück-Sortieranlage erstellen & 10,0 Stunden\\
        \hline
        Abnahmetest erstellen & 8,0 Stunden\\
        \hline
        Dokumentation vervollständigen (inkl. „Lessons learned“) & 9,0 Stunden\\
        \hline
      \end{tabular}
    \end{center}
  \end{small}
\end{document}